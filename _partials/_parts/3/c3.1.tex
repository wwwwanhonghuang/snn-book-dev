
\chapter{Quantum Neuromorphic Computing}
\section{Taxonomy} \index{Taxonomy for Quantun Neuromorphic Computing}


\begin{table}[h!]
    \centering
    \begin{tabular}{|p{3.5cm}|p{2.5cm}|p{4cm}|p{4cm}|} % Adjust column width for automatic line breaking
    \hline
    \textbf{Simulation Method} & \textbf{Philosophical Category} & \textbf{Characteristics} & \textbf{Example Application} \\
    \hline
    Element-Microscopic (Particle-Level) & Elementarism & Focuses on simulating fundamental components (e.g., particles, neurons) individually, often with reductionist assumptions. & Modeling individual qubits or synapses in Quantum SNNs using Schrödinger's equation. \\
    \hline
    System-Macroscopic (Whole-System) & Holism & Treats the system as a whole, capturing collective dynamics and emergent properties without decomposing into parts. & Analyzing the collective state of a Quantum SNN with wave functions. \\
    \hline
    Operational (Circuit/Annealing) & Operationalism & Focuses on practical, measurable implementations using existing technologies, like quantum circuits or annealing. & Implementing SNNs on quantum hardware such as Qiskit or D-Wave. \\
    \hline
    Field-Based Modeling & Field Theory & Treats phenomena as continuous fields rather than discrete entities; useful for spatial-temporal dynamics. & Modeling neural activity as a continuous field in the brain using PDEs. \\
    \hline
    Hybrid Quantum-Classical Simulations & Pragmatism & Combines quantum and classical techniques, prioritizing practical performance over theoretical purity. & Using hybrid quantum-classical algorithms for optimizing neural networks. \\
    \hline
    Topological Methods & Structuralism & Focuses on the structure of data; uses shapes and topological features to extract information from high-dimensional spaces. & Analyzing EEG microstates using persistent homology and topological features. \\
    \hline
    Dynamical Systems Modeling & Process Philosophy & Models systems based on continuous evolution over time, emphasizing processes and change. & Using differential equations to simulate spiking dynamics in neurons. \\
    \hline
    Generative Quantum SNNs & Generativism / Constructivism & Uses generative grammar and principles to model and generate/construct the evolution and dynamics of quantum spiking neural networks. & Applying generative models to simulate quantum states and guide the evolution of quantum SNNs. \\
    \hline
    \end{tabular}
    \caption{Philosophical Taxonomy of Simulation Methods for Quantum SNNs}
    \label{tab:taxonomy_simulation_methods}
    \end{table}
    

\section{System Dynamics-Based Methods}\index{System Dynamics-Based Methods}
\section{Neuron Simulation Methods}\index{Neuron Simulation Methods}

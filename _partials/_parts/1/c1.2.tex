
%%\chapterimage{ima2} % Chapter heading image

\chapter{Neuron Dynamics}\

Spiking neural network's neuron model describe how the membrance potential of a neuron change over the time.
In this chapter, we dive into the biological neuron's dynamics and exhibit several examples neuron models in the 
spiking neural network. 

% Beside classical neuron models such as 
% \texttt{Hodgkin-Huxley (HH) model}, \texttt{Leaky integrate-and-fire model}, and \texttt{Zhikevich model},
% we also describe some 

\section{Biological Neuron}\index{Biological Neuron}


\section{Neuron Model Abstract and Taxonomy}\index{Neuron Model Abstract and Taxonomy}
In spiking neural network, Equation~\ref{eq:neuron-state-space-abstract-1} to
Equation~\ref{eq:neuron-state-space-abstract-4} provide a highly abstract description for
 modeling a biological neuron.
In these equation, $\bold x_t$ is the state vector at time $t$, $\Delta x_t$ is the state variation at time 
$t$, $\bold u_t$ is the input vector at time $t$, $V$ is the membrane potential of the neuron, 
$\bold y_t$ is the neuron's output at time $t$, 
and $V_{out}$ is the output voltage that will be sent to the synapses that depature from this neuron. 
$(\bold x_{t})_{i}$ is the $i$-th element of the state vector.

\begin{align}
    \Delta \bold x_t = f(\bold x_t, \bold u_t) \label{eq:neuron-state-space-abstract-1} \\
    \bold y_t = g(\bold x_t, \bold u_t) \label{eq:neuron-state-space-abstract-2} \\
    s.t.,\; (\exists i \in [0, |\bold x_t|])(\bold x_{t})_{i} = V \label{eq:neuron-state-space-abstract-3}\\
            (\exists i \in [0, |\bold y|])_i\bold y = V_{out} \label{eq:neuron-state-space-abstract-4}
\end{align}

These equations are in the form of a \textbf{state-space model (SMM)}.


Equation~\ref{eq:neuron-state-space-abstract-3} indicates that, a neuron should maintain a membrane potential $V$, 
and Equation~\ref{eq:neuron-state-space-abstract-4} indicates that, the neuron's output should contains a 
voltage $V_{out}$.

\subsection{Current-based Neuron}\index{Current-based Neuron}
In a current-based neuron model, the synaptic input is represented as an injected current directly 
added to the membrane potential equation.
we have the input $\bold u_t$ contains the current $I_t$, and $\bold x_t$ contains the membrance potential $V_t$.

We will later to see that a current-based LIF neuron (Section~\ref{sec:current-based-lif}) can hold a form of 
Equation~\ref{eq:current-based-lif-state-space}.


\begin{equation}\label{eq:current-based-lif-state-space}
    \dot V_t = \frac{1}{\tau_m}(-(V_t-V_{rest})+I_t) = f_{CurrentLIF_1}(V_t, \bold u_t = [I_t])
\end{equation}



\subsection{Conductance-based Neuron}\index{Conductance-based Neuron}
In a conductance-based neuron model, the synaptic input is modeled by changing the conductance of the membrane, 
which then affects the current flow.

In the state space representation, the input $\bold u_t$ is the synapse conductance.
let $\bold u_t=g_{syn}(t)$, $\bold x_t = V_t$, we will see that in conductance-based LIF neuron model, it 
hold Equation~\ref{eq:condunce-based-lif-state-space}

\begin{equation}\label{eq:condunce-based-lif-state-space}
    \dot V_t = \frac{1}{\tau_m}(-(V_t-V_{rest})+g_syn(t)(E_{syn}-V_t)) = f_{ConductanceLIF_1}(V_t,  \bold u_t = g_{syn}(t))
\end{equation}




\section{Neuron Model Examples}\index{Neuron Model Examples}
\subsection{Hodgkin-Huxley (HH) Model}\index{Hodgkin-Huxley (HH) Model}

The Hodgkin-Huxley (HH) model is a conductance-based model , which can be utilize to accurately 
reproduce the bio-neuron's dynamics. Its form is shown in Equation~\ref{eq:hh-model-1} to Equation~\ref{eq:hh-model-5}.
Combine all these equations, we get Equation~\ref{eq:hh-model-6}.
% \begin{align} 
% \begin{split}
% P(y=1/x)   &=P(y*>0/x)=P[x\beta+e>0/x]=P[e>-(\beta_{0}+x\beta)/x]\\
% &=1-F[-(\beta_{0}+x\beta)]=F(\beta_{0}+x\beta)
% \end{split}					
% \end{align}

\begin{align}
    I = C_m\frac{dV_m}{dt} + I_{i} \label{eq:hh-model-1} \\
    I_i = I_{Na} + I_K + I_l \label{eq:hh-model-2}  \\
    I_{Na} = g_{Na}(V_m - V_{Na}) \label{eq:hh-model-3} \\
    I_K = g_{K}(V_m - V_{K}) \label{eq:hh-model-4}  \\
    I_l = \bar g_{l}(V_m - V_{l}) \label{eq:hh-model-5} 
\end{align}

\begin{equation}\label{eq:hh-model-6}
    I_l = C_m\frac{dV_m}{dt} + g_{Na}(V_m - V_{Na}) + g_{K}(V_m - V_{K}) + \bar g_{l}(V_m - V_{l})
\end{equation}

Ion channel function $g_{\cdot}$ are function respect to time $t$ and membrance potential $V$.
Specifically, Equation~\ref{eq:hh-model-7} is held.

\begin{equation}\label{eq:hh-model-7}
    g_{Na} = \bar g_{Na}m^3h \\
    g_{K} = \bar g_{K}n^4 \\
    g_{l} = \bar g_l \\
\end{equation}

Combine Equation~\ref{eq:hh-model-1} to Equation~\ref{eq:hh-model-7}, we get Equation~\ref{eq:hh-model-8}
\begin{equation}\label{eq:hh-model-8}
    I_l = C_m\frac{dV_m}{dt} + \bar g_{Na}m^3h(V_m - V_{Na}) + \bar g_{K}n^4(V_m - V_{K}) + \bar g_{l}(V_m - V_{l})
\end{equation}

$\frac{d\cdot}{dt}= \alpha_\cdot(V_m)(1-\cdot)-\beta_\cdot(V_m)\cdot $ is held. Where $\cdot$ is a placeholder for
$m$, $n$ and $h$. As such, Equation~\ref{eq:hh-model-9} to Equation~\ref{eq:hh-model-11} are held.

\begin{align}
    \frac{dn}{dt}= \alpha_n(V_m)(1-n)-\beta_n(V_m)n \label{eq:hh-model-9}\\
    \frac{dm}{dt}= \alpha_m(V_m)(1-m)-\beta_m(V_m)m \label{eq:hh-model-10}\\
    \frac{dh}{dt}= \alpha_h(V_m)(1-h)-\beta_h(V_m)h \label{eq:hh-model-11}\\
\end{align}

From experiment, 
we have Equation~\ref{eq:hh-model-12} to Equation~\ref{eq:hh-model-17}.

\begin{align}
    \alpha_n(V_m) = \frac{0.01(10-V)}{exp(\frac{10-V}{10}) - 1} \label{eq:hh-model-12}\\
    \alpha_m(V_m) = \frac{0.1(25-V)}{exp(\frac{25-V}{10}) - 1} \label{eq:hh-model-13}\\
    \alpha_h(V_m) = 0.07exp(-\frac{V}{20}) \label{eq:hh-model-14}\\
    \beta_n(V_m) = 0.125exp(-\frac{V}{80}) \label{eq:hh-model-15}\\
    \beta_m(V_m) = 4exp(-\frac{V}{18}) \label{eq:hh-model-16}\\
    \beta_h(V_m) = \frac{1}{exp(\frac{30-V}{10})+1} \label{eq:hh-model-17}
\end{align}

Hodgkin-Huxley could be seen as a current-based neuron model, which may represent by space state model,
with $\bold x_t = [V_t, m_t, h_t, n_t]$, and $\bold u_t = I_t$.

\subsection{Leaky Integrate-and-fire Model}\index{Leaky Integrate-and-fire Model}
\label{sec:current-based-lif}
Leaky Integrate-and-fire model is a computational effective model, 
in which a threshold is set, when membrance potential cross the threshold, 
the neuro emit a spike. 
In implementation, we may set the voltage output at time $t$, $V_{out}$ be $1\;mV$. 
The scale problem has the potential to be solved by automatically adjusting the synapses' weights in trainning.
A LIF neuron $i$'s output form a \textbf{spike train} $S_{i}(t) = \sum_{m}\delta(t-t_m)$.

\begin{equation}\label{eq:lif}
    \dot V_t = \frac{1}{\tau_m}(-(V_t-V_{rest})+g_syn(t)(E_{syn}-V_t))
\end{equation}

%% Learning neuron, synapse models and the spike trainning method.
%% try on real dataset.

\subsection{Izhikevich Model}\index{Izhikevich Model}

% \subsection{Introducción}\index{Introducción}

\subsection{FitzHugh-Nagumo Model}\index{FitzHugh-Nagumo Model}

\subsection{Morris-Lecar Model}\index{Morris-Lecar Model}

\subsection{Hindmarsh-Rose Model}\index{Hindmarsh-Rose Model}

\subsection{Cable theory}\index{Cable theory}

\subsection{Perfect Integrate-and-fire}\index{Perfect Integrate-and-fire}


\subsection{Adaptive Integrate-and-fire}\index{Adaptive Integrate-and-fire}

\subsection{Fring Rate Model}\index{Firing Rate Model}

\subsection{Discussion}\index{Discussion (Neuron Dynamics)}

\textbf{Spike Representation}
You may note that, although different neuron models have different dynamics and spike 
representations, they can still communicate with each other through synapses. Some neuron
 models require current input, while others do not. Nonetheless, spikes can transmit 
 over synapses and cause current variations across them by utilizing a common output element $V_{out}$. 
 
 One challenge that may arise is the normalization of spike representations.
To address this issue, one approach is to standardize the spike events by converting all
 spike representations into binary form. Another approach can involve adjusting the synaptic weights.
  Although we may face scaling issues with different types of spike representations, these
   can be mitigated during training by appropriately adjusting the synapse weights.



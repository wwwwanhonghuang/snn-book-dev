
\chapter{Synapse Dynamics}
\section{Biological Synapse}\index{Biological Synapse and Taxonomy}

\section{Synapse Abstract and Taxonomy}
Similar to neuron dynamics, the synapse dynamics could be modeled a state space model,
and the current $I$ is included in the state $\bold x_t$. We only need modified the 
constraint term, as shown in Equation~\ref{eq:synapse-state-space-abstract-1} to 
Equation~\ref{eq:synapse-state-space-abstract-4}. 

% \begin{align}
%     \Delta \bold x_t = f(\bold x_t, \bold u_t) \label{eq:synapse-state-space-abstract-1} \\
%     \bold y_t = g(\bold x_t, \bold u_t) \label{eq:synapse-state-space-abstract-2} \\
%     s.t.,\; (\exists i \in [0, |\bold x(t)|])(\bold x_{t})_{i} = I_t \label{eq:synapse-state-space-abstract-3}\\
%             (\exists i \in [0, |\bold y(t)|])_i\bold y = I_{out} \label{eq:synapse-state-space-abstract-4}
% \end{align}

The synapse model can be categorized to current-based synapses and 
conductance-based synapses. Both of them can be seems as a 
convolution operations on the generalized weighted preneuron spike train.
Specifically, we may write $I(t)=(\theta\cdot S(t)) * h(t)$, where $\theta\triangleq \{w_i, g_{max}\}$, 
, $h(t)$ is a convolution kernel, and $S(t)$ is the pre-synapse neuron's spike train.


\subsection{Current-based Synapse}
% In current-based synapses, the postsynaptic effect is modeled as an 
% instantaneous current added to the postsynaptic neuron.
% Example: Synaptic current as an exponential decay
% \begin{equation}
% I_{syn}(t)=I_{peak}e^{-(t-t_{spike})/\tau_s}
% \end{equation}
% $I_{peak}$: peak current. $t_{spike}$: the time of the presynaptic spike. 
% $\tau_s$:  the time constant of the synaptic current.

% In the context of synaptic models, the $V$ in the synaptic current equation 
% generally refers to the membrane potential of the postsynaptic neuron

\subsection{Conductance-based Synapse}
% E.g., Alpha function synapse:
% $g_{syn}(t)=g_{max}\frac{t-t_{spike}}{\tau}e^{-\frac{t-t_{spike}}{\tau}}$
% The synaptic current then becomes:
% $I_{syn}(t)=g_{syn}(t)(E_{syn}-V(t))$

\subsection{Chemical Synapse}
% Chemical synapses can be modeled using either current-based or conductance-based approaches,
%  but these are modeling choices rather than fundamentally different categories. 

 \textbf{Current-based Synapse}

%  Modeling Approach: In a current-based synapse model, the effect of neurotransmitter release 
%  is represented as an injected current directly added to the postsynaptic
%   neuron's membrane potential.

\textbf{Conductance-based Synapse}

%   Modeling Approach: In a conductance-based synapse model, the effect of neurotransmitter 
%   release is modeled by changing the synaptic conductance, which then affects the current flow based 
%   on the difference between the membrane potential and the synaptic reversal potential.

\section{Discussion}
% In a summary, all kind of neuron should maintains a state of membrance $V$, its variation can in a 
% current-based method or a condunctance-based method. The variation of $V$ can make the current changed 
% in synapse. And current $I$ is a nessary state maintained by synapses no matter in what kind of methods.
% Though in a condunctance-based method, we can retrieve the synapse $I$ from equations. 



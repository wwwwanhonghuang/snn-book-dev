
\chapter{Synapse Dynamics}
\section{Biological Synapse}\index{Biological Synapse and Taxonomy}

\section{Overview of Synapse Models and Taxonomy}\index{Overview of Synapse Models and Taxonomy}
Similar to neuron dynamics, the synapse dynamics could be modeled utilize a state space model,
and the current term $I(t)$ is included in the state $\bold x(t)$. We only need modified the 
constraint term, as shown in Equation~\ref{eq:synapse-state-space-abstract-1} to 
Equation~\ref{eq:synapse-state-space-abstract-3}. 

\begin{align}
    \bold{\dot x}(t) = f(\bold x(t), \bold S(t)) \label{eq:synapse-state-space-abstract-1} \\
    \bold y(t) = g(\bold x(t)) \label{eq:synapse-state-space-abstract-2} \\
    s.t.,\; (\exists i \in [0, |\bold x(t)|])(\bold x_{t})_{i} = I(t) \label{eq:synapse-state-space-abstract-3}\\
\end{align}

The synapse model can be categorized to current-based synapses and 
conductance-based synapses. 
The model that directly model the synapse current $I(t)$, it is also known as \textbf{current-based synpase model}.
On the other hand, when we model the \textbf{synaptic kinetics} $s_{syn}(t)$ of a synpase and retrieve the 
synpase current by $I(t) = g_{syn}s_{syn}(t)(V_{syn} - V_m(t))$, the synpase model is also known as 
\textbf{conductance-based synpase}.
  
Both of current and synaptic kinetics can be represented as a convolution operations on
 the generalized weighted preneuron spike train.
Specifically, we may write $z(t)=(\theta\cdot S(t)) * h(t)$, where $o\triangleq \{I, s_{syn}\}$, 
, $h(t)$ is a convolution kernel, and $S(t)$ is the pre-synapse neuron's spike train. $z(t)$ is 
the generalized output, which is the current $I$ when it is a current-based synapse model, 
while $s_{syn}$ when it is a conductance-based synapse model.


\subsection{Current-based Synapse}\index{Current-based Synapse}
% In current-based synapses, the postsynaptic effect is modeled as an 
% instantaneous current added to the postsynaptic neuron.
% Example: Synaptic current as an exponential decay
% \begin{equation}
% I_{syn}(t)=I_{peak}e^{-(t-t_{spike})/\tau_s}
% \end{equation}
% $I_{peak}$: peak current. $t_{spike}$: the time of the presynaptic spike. 
% $\tau_s$:  the time constant of the synaptic current.

% In the context of synaptic models, the $V$ in the synaptic current equation 
% generally refers to the membrane potential of the postsynaptic neuron

\subsection{Conductance-based Synapse}\index{Conductance-based Synapse}
% E.g., Alpha function synapse:
% $g_{syn}(t)=g_{max}\frac{t-t_{spike}}{\tau}e^{-\frac{t-t_{spike}}{\tau}}$
% The synaptic current then becomes:
% $I_{syn}(t)=g_{syn}(t)(E_{syn}-V(t))$

\subsection{Chemical Synapse}\index{Chemical Synapse}
% Chemical synapses can be modeled using either current-based or conductance-based approaches,
%  but these are modeling choices rather than fundamentally different categories. 

 \textbf{Current-based Synapse}

%  Modeling Approach: In a current-based synapse model, the effect of neurotransmitter release 
%  is represented as an injected current directly added to the postsynaptic
%   neuron's membrane potential.

\textbf{Conductance-based Synapse}

%   Modeling Approach: In a conductance-based synapse model, the effect of neurotransmitter 
%   release is modeled by changing the synaptic conductance, which then affects the current flow based 
%   on the difference between the membrane potential and the synaptic reversal potential.

\section{Discussion}\index{Discussion (Synapse Models)}
% In a summary, all kind of neuron should maintains a state of membrance $V$, its variation can in a 
% current-based method or a condunctance-based method. The variation of $V$ can make the current changed 
% in synapse. And current $I$ is a nessary state maintained by synapses no matter in what kind of methods.
% Though in a condunctance-based method, we can retrieve the synapse $I$ from equations. 


